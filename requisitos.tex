\documentclass[a4paper]{article}
\usepackage[utf8]{inputenc}
\usepackage[brazil]{babel}
\usepackage{pdfpages}
\usepackage{enumitem}
\usepackage{graphicx}
\usepackage[hmargin=2cm,vmargin=1cm,bmargin=2cm]{geometry}

\newcommand{\horrule}[1]{\rule{\linewidth}{#1}} 

%\renewcommand{\labelenumi}{\Alph{enumi}.}
%\renewcommand{\labelenumi}{\Alph{enumi}.\arabic{enumii}}

\title{
\normalfont \normalsize \Huge{\textsc{Sistema para gerenciamento de biblioteca}}
\horrule{0.2pt} 
}
\author{\textsc{Análise e Projeto de Software Orientado a Objetos - Grupo 4}}

\begin{document}
	\maketitle
	\date
	\section*{Descrição do problema:}
	\paragraph{}
	A Pós-Graduação da Facom (Faculdade de Computação da UFMS) possui uma pequena biblioteca de livros, dissertações e teses. Esse acervo, de acesso exclusivo a alunos de pós-graduação e professores tem problemas de divulgação e de controle de empréstimo. Assim, são reportados subuso e alguma dificuldade no controle exato de acervo emprestado, devolvido ou desaparecido. Surgiu então a ideia de informatizar o processo.
	\paragraph{}
	Para ter acesso à biblioteca, o usuário - aluno de pós-graduação ou professor - deve ter um cadastro com nome completo, endereço de e-mail, CPF e telefone. O usuário pode alterar senha, e-mail, telefone e foto de seu perfil identificador quando quiser. Quando um aluno ou professor não é mais da pós-graduação da Facom, ele não pode mais emprestar títulos da biblioteca.
	\paragraph{}
	O acervo da biblioteca é dividido em livros, dissertações e teses. Livros são identificados pelo título, editora, ISBN, autores e descrição. A origem do livro, isto é, se foi doado, comprado por fundação de pesquisa etc deve ser registrado. Dissertações e teses são identificadas por nome do aluno, do orientador, título, palavras-chave, resumo, ano e área de pesquisa. É preciso manter também informação de edição, volume, ano e número de exemplares. Para facilitar a busca de acervo, uma foto da capa deve ser divulgada, sempre que possível. Mesmo que um título deixe de fazer parte do acervo, um registro deve ser mantido.
	\paragraph{}
	É preciso divulgar o acervo na internet, para que os usuários saibam quais títulos eles podem emprestar. Saber a quantidade disponível de cada título é importante para os professores, quando vão utilizá-los como referencial bibliográfico de suas aulas. Também deve ser possível reservar um título que o usuário queira emprestar, caso esteja com outro usuário ou caso queira garantir um empréstimo. As reservas devem ter um prazo de 3 dias de duração. Após esse prazo, elas perdem o valor.
	\paragraph{}
	Ao buscar um título, é necessário divulgar título, autor, edição, volume, ano de publicação, número de exemplares e seu estado atual, isto é, se está disponível, reservado ou emprestado. Somente títulos disponíveis e emprestados pode ser reservados.
	\paragraph{}
	Professores podem emprestar títulos pelo prazo de 60 dias e alunos, 30, podendo ser renováveis por igual período enquanto não houver reserva para o título. O usuário deve ser notificado sobre seus empréstimos, suas reservas e seus prazos de devolução. Caso um usuário esteja devendo a devolução de um título, ele deve ser notificado todos os dias, não podendo realizar novas reservas ou novos empréstimos.
	\paragraph{}
	Relatórios de informação de acervo, usuários e acessos são básicos para a manutenção da biblioteca.
	\paragraph{}
	A página inicial deve ser fácil de usar e de entender.
	\section*{A. Visão geral do sistema}
	\paragraph{}
	O software de apoio à biblioteca tem como objetivo principal auxiliar o gerenciamento de empréstimos de títulos de livros, dissertações e teses. Os usuários do software são pessoas interessadas em conhecer e manter controle sobre o acervo. O software deverá permitir o cadastro de pessoas usuárias da biblioteca, de títulos do acervo, e registrar reservas, empréstimos, renovações e notificações.
	
	\section*{B. Requisitos funcionais}
	B.1 Cadastro de usuários
		\begin{enumerate}
			\item O sistema deve permitir o cadastro de pessoas usuárias da biblioteca. As informações a serem cadastradas são: nome completo, e-mail, CPF, telefone para contato e senha. No cadastro, deve-se especificar se o usuário é do tipo aluno, professor, administrador ou secretário.
			%\item O sistema deve permitir que os usuários alterem ou recuperem suas senhas e telefones de contato. Os usuários podem selecionar uma foto que o identifique.
			\item O sistema deve permitir que o administrador do sistema possa cadastrar e alterar todos os dados de cadastro, incluindo se um usuário está ativo ou não no sistema, menos a senha pessoal de acesso de cada usuário. Somente usuários ativos podem emprestar livros.
			\item O sistema deve registrar, para usuários do tipo aluno, o prazo de conclusão do curso, para inativar automaticamente seu cadastro após tal prazo.
		\end{enumerate}
	B.2 Cadastro de livros
		\begin{enumerate}[resume]
			\item O sistema deve permitir o cadastro de títulos do acervo. Os títulos são classificados em livros, dissertações e teses. Livros possuem título, editora, ISBN, autores e descrição. A origem do livro, isto é, se foi doado, comprado por fundação de pesquisa etc deve ser registrado. Dissertações e teses são identificadas por nome do aluno, do orientador, título, palavras-chave, resumo, ano e área de pesquisa.
			\item O sistema deve manter também informação de edição, volume, ano e número de exemplares. Para facilitar a busca de acervo, uma foto da capa deve ser divulgada, sempre que possível.
			\item O sistema deve manter um histórico permanente de todos os títulos já catalogados.
		\end{enumerate}
	B.3 Realização de uma busca no acervo
		\begin{enumerate}[resume]
			\item O sistema deve permitir que todo usuário, cadastrado ou não no sistema, possa consultar o acervo e saber se determinado título está disponível para empréstimo, reservado ou emprestado.
			\item O sistema deve informar título, autor, edição, volume, ano de publicação, número de exemplares e seu estado atual, isto é, se está disponível, reservado ou emprestado. Um contato de e-mail do usuário que esteja com um livro buscado emprestado.
		\end{enumerate}
	B.4 Reserva de um título
		\begin{enumerate}[resume]
			\item O sistema deve permitir que todo usuário que acessou o sistema e efetuou uma busca no acervo possa reservar um título, caso este esteja emprestado ou disponível.
			\item O sistema deve notificar o usuário sobre o prazo de validade de sua reserva, que é de 3 dias úteis. Uma reserva expirada também deve ser comunicada ao usuário, tornando o título disponível para empréstimo.
		\end{enumerate}
	B.5 Empréstimo de um título
		\begin{enumerate}[resume]
			\item O sistema deve permitir o registro de empréstimo de títulos. Professores podem emprestar títulos pelo prazo de 60 dias e alunos, 30, renováveis enquanto não houver reserva para o título.
			\item O sistema deve notificar o usuário sobre a confirmação de um empréstimo, informando-lhe o prazo para devolução.
			\item O sistema deve notificar um usuário que esteja devendo a devolução de um título.
			\item O sistema deve bloquear novas reservas e novos empréstimos de usuários com pendências de devolução.
		\end{enumerate}
	B.6 Renovação de empréstimo
		\begin{enumerate}[resume]
			\item O sistema deve permitir a renovação de um título emprestado por período igual ao do prazo de empréstimo.
			\item O sistema deve notificar o usuário sobre uma renovação confirmada, registrando o novo prazo de empréstimo.
		\end{enumerate}
	B.7 Devolução de um título
		\begin{enumerate}[resume]
			\item O sistema deve registrar uma devolução de título emprestado, registrando a quantidade de dias que o título permaneceu com o usuário, e a quantidade de dias em atraso de devolução.
			\item O sistema deve notificar o usuário sobre devoluções confirmadas, tornando o título disponível para empréstimo.
		\end{enumerate}
	B.8 Emissão de relatórios
		\begin{enumerate}[resume]
			\item O sistema deve emitir relatório com dados do acervo, classificados por tipo e demais dados.
			\item O sistema deve emitir relatório com dados de usuário, classificados por tipo e demais dados, calculando a quantidade de dias em atraso de devolução de cada um.
			\item O sistema deve emitir relatório com o registro de acesso de usuários, com data, horário e tipo de ação executada.
		\end{enumerate}
	\section*{C. Requisitos não funcionais}
	\paragraph{}
	Usabilidade
	\begin{enumerate}[resume]
		\item O sistema deve ser fácil de usar e considerar as sub-características de usabilidade apresentadas na ISO/IEC 9126.
	\end{enumerate}
	\paragraph{}
	Confiabilidade
	\begin{enumerate}[resume]
		\item O sistema deve ter capacidade para recuperar dados perdidos da última operação realizada em caso de falha.
		\item O sistema deve estar disponível com o mínimo de interrupções possível.
	\end{enumerate}
	\paragraph{}
	Portabilidade
	\begin{enumerate}[resume]
		\item O  sistema  deve  ser  executado  em  computadores  Intel\textsuperscript{TM} Core 2 Duo  ou  superior,  em qualquer sistema operacional.
	\end{enumerate}
	\newpage
	\section*{D. Glossário}
	\begin{enumerate}
		\item \textbf{Acervo:} Conjunto dos livros, dissertações e teses da biblioteca.
		\item \textbf{Consulta:} Realização de uma busca por determinado título no acervo na biblioteca, via interface online.
		\item \textbf{Devolução:} Registro de devolução do título emprestado pelo usuário.
		\item \textbf{Empréstimo:} Retirada de determinado título, com compromisso de devolução no prazo especificado.
		\item \textbf{ISBN:} International Standard Book Number, é um sistema que identifica numericamente os livros segundo o título, o autor, o país e a editora, individualizando-os inclusive por edição.
		\item \textbf{Notificação:} Mensagens enviadas pelo sistema ao e-mail do usuário, registrando cadastro de usuário, devolução, empréstimo, renovação e reserva de título.
		\item \textbf{Renovação:} Repetição do prazo de devolução do título, desde que o título não possua reserva registrada em nome de outro usuário.
		\item \textbf{Reserva:} Direito temporário sobre empréstimo de determinado título, assim que ele é devolvido ao acervo.
		\item \textbf{Restrição:} Impedimento do usuário para executar empréstimo, renovação e reserva de título, que vigora até o momento da devolução do título com prazo de devolução vencido.
		\item \textbf{Usuário:} Pessoas que podem realizar empréstimos da biblioteca, compreendendo alunos de pós-graduação e professores da Facom.
	\end{enumerate}
\end{document}
